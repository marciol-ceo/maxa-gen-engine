\section*{Exercice 1}

Dans une compétition de tir à l'arc, des numéros de participation de 1 à $m$ ont été attribués aléatoirement à $m$ participants, qui tireront dans l'ordre des numéros. Deux frères, $C$ et $D$, participent également. On note $n_{C}$ et $n_{D}$ leurs numéros respectifs.
\begin{enumerate}
\item Combien y a-t-il de paires $\left(n_{C}, n_{D}
\right)$ possibles?
\item Soit $s$ un entier tel que $1 \leq s \leq m-1$. Montrer que $2(m-s)$ des paires de la question précédente vérifient $\left|n_{C}-n_{D}\
ight|=s$.
\item Quel est l'écart le plus probable entre $n_{C}$ et $n_{D}$ ?
\item Quel est l'écart moyen entre $n_{C}$ et $n_{D}$ ?
\end{enumerate}